\chapter{Indledning}
Tal fra Hjerteforeningen* siger, at der i Danmark i 2014 blev registreret 159.561 patienter med åreforsnævring af hjertets koronararterier, heraf havde 53.117 af patinterne et tilfælde af akut blodprop i hjertet. Akut blodprop i hjertet, eller akut myokardieinfarkt, inddeles i STEMI eller Non-STEMI, ved begge tilfælde er det afgørende for patientens overlevelse at diagnose og behandling forekommer så hurtigt som muligt efter proppens indtrædelse.
Operationer, nogle former for medicinsk behandling og de generelle bekymringer ved at være indlagt på et hospital, disponerer for hjerte-kar-sygdomme, fordi det øgede stress på kroppen kan medføre forhøjet blodtryk, hvilket er en af det største risikofaktorer for udvikling af hjerte-kar-sygdomme.
For at hjælpe læger og andet sundhedspersonale til at opdage tilfælde af STEMI og Non-STEMI inden det er for sent, og dermed mindste antallet af dødsfald ved disse sygdomme har vi udviklet G4-EKG, et EKG-monitoreringssystem, med speciale i at detektere og alamere sundhedspersonale ved tilfælde af akut myokardieinfarkt på hospitalerne. 
G4-EKG består af et hovedsystem (UIKontor) som implementeres på afdelingskontoret. Herpå kan man se data for op til 6 patienter af gangen, herunder de mest primære patientdata (navn, CPR, indlæggelsesdata, stue nr.) og udskrifter af tidligere foretagne EKG-målinger for den pågældende patient. Det vil desuden være på UIKontor, at man starter og stopper EKG-målinger af patinter på de forskelligt tilkoblede stuer. 
Yderligere findes der i G4-EKG en UIPatient, som er den grafiske fremstilling af EKG-målingerne ude på stuerne. UIPatient indeholder altså en graf over den nuværende patients EKG, patientens HRV (Heart Rate Variation), patientens nuværende puls og desuden en boks der kan give indikationer på om der forefindes STEMI eller Non-STEMI hos patienten. Den endelige diagnose lades være op til lægen selv, som enten kan stille diagnosen ud fra live-feedet på UIPatient eller ud fra en udskrift af patientens EKG, som han/hun kan finde på UIKontor. 
G4-EKG har gode muligheder for udbyggelse, man kan bl.a. tilføje flere patienter til UIKontor, hvis man har behov for det. En oplagt udvidelse ville være at give systemet kode til at kunne detektere flere arytmier, som fx arterieflimren. 
Følgende rapport beskriver G4-EKG’s opbyggelse, beskrevet gennem en kravspecifikation, systemarkitektur, systemdesign og en test af hele systemet. 